% \subsection{Stock Price Prediction}
% Stock Price Prediction is the task of forecasting future stock prices based on historical data and various market indicators. 
% It involves the use of statistical models and machine learning algorithms to analyze financial data and make predictions about a stock's future performance. 
% Stock price trends are influenced by numerous factors, including interest rates, inflation rates, and financial news. 
% To make accurate stock price predictions, one must leverage this diverse set of information. 
% In the banking industry and financial services sector, teams of analysts are dedicated to scrutinizing, analyzing, and quantifying qualitative data from news sources. 
% A substantial amount of information regarding stock trends is extracted from the extensive corpus of textual and quantitative data involved in such analysis.

Recent research trends in stock price prediction include advancements in deep learning-based regression models. \cite{li2019dp}
These models often utilize Long Short-Term Memory (LSTM) networks and innovative validation techniques, such as walk-forward validation, to enhance their predictive capabilities.

In addition, some researchers have explored Particle Filter Recurrent Neural Networks (PF-RNNs), a new RNN family explicitly designed to model uncertainty within their internal structure. \cite{ma2020particle}
Unlike traditional RNNs that rely on a deterministic latent state vector, PF-RNNs maintain a latent state distribution approximated as a set of particles. 
To enable effective learning, researchers have introduced a fully differentiable particle filter algorithm that updates the PF-RNN latent state distribution based on Bayes' rule. 
Experimental results have shown that PF-RNNs can outperform conventional gated RNNs across various domains, 
including synthetic robot localization datasets and real-world sequence prediction tasks, which is stock price prediction.

Furthermore, recent studies have proposed novel approaches, such as the development of a sentiment-ARMA model, which combines the autoregressive moving average model (ARMA) with sentiment analysis of financial news articles. \cite{mehtab2021stock}
This model is integrated into an LSTM-based deep neural network consisting of three components: LSTM, VADER model, and a differential privacy (DP) mechanism. 
The proposed DP-LSTM scheme has demonstrated the potential to reduce prediction errors and enhance model robustness. Extensive experiments conducted on S\&P 500 stocks have indicated promising results, 
including a 0.32\% improvement in mean Mean Percentage Absolute Error (MPA) 
and a significant up to 65.79\% reduction in Mean Squared Error (MSE) for the prediction of the market index S\&P 500.

