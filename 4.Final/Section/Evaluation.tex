\subsection{Evaluation Metric}

\subsubsection{Mean Prediction Accuracy (MPA)}
To assess the effectiveness of our methods, we calculate the Mean Prediction Accuracy (MPA), defined as:
\begin{align}
    \label{eq:MPA}
    \text{MPA}_{t} = 1 - \frac{1}{N} \sum_{i=1}^{N} \frac{|\hat{y}_i - y_i|}{y_i}
\end{align}
where $ N $ is the number of stocks, $ \hat{y}_i $ and $ y_i $ represent the predicted and actual high or low prices, respectively, for the $ t $-th day of stock $ i $.

\subsubsection{Mean Absolute Error (MAE)}
The Mean Absolute Error (MAE) is another metric used to evaluate our methods:
\begin{align}
    \label{eq:MAE}
    \text{MAE}_{t} = \frac{1}{N} \sum_{i=1}^{N} |\hat{y}_i - y_i|
\end{align}
Here, $ N $ denotes the number of stocks, with $ \hat{y}_i $ and $ y_i $ being the predicted and actual high or low prices for the $ t $-th day.

\subsubsection{Trend Accuracy (TAC)}
Given our focus on short-term stock price trends, we also compute the Trend Accuracy (TAC):
\begin{align}
    \label{eq:TAC}
    \text{TA}_{t} = \frac{1}{N} \sum_{i=1}^{N} \mathbb{1}(\text{sign}(\hat{y}_i - \hat{y}_{t-1}) = \text{sign}(y_i - y_{t-1}))
\end{align}
where $ N $ is the number of stocks, $ \hat{y}_i $ and $ y_i $ are the predicted and actual high or low prices for the $ t $-th day, and $ \mathbb{1}(\cdot) $ is the indicator function.

\subsubsection{Accuracy}
To compare the similarity between the predicted and actual price intervals, we calculate the accuracy(ACC):
\begin{align}
    \label{eq:accuracy}
    \text{Accuracy}_{t} = \frac{1}{N} \sum_{i=1}^{N} \frac{\text{length}(\{ y_{\min}<y<y_{\max} \cap \hat{y}_{\min}<y<\hat{y}_{\max} \})}{\max(\text{length}(\hat{y}_{\min}<y<\hat{y}_{\max}), \text{length}(y_{\min}<y<y_{\max}))}
\end{align}
where $ N $ is the number of stocks, $ \hat{y}_{\min} $ and $ \hat{y}_{\max} $ are the predicted minimum and maximum prices for the $ t $-th day, and $ y_{\min} $ and $ y_{\max} $ are the actual minimum and maximum prices for the $ t $-th day.

\subsection{Result}


\begin{table}[h]
	\label{tab:result}
	\centering
	\begin{tabular}{c|c|c|c|c|c|c|c|c|c}
	\toprule\toprule
	% \multirow{2}{*}{\textbf{Model}}
	\multicolumn{2}{c}{}& \multicolumn{4}{|c|}{\textbf{STABLE}} & \multicolumn{4}{c}{\textbf{UNSTABLE}} \\
	\midrule
	\multicolumn{2}{c|}{\textbf{Model}} & \textbf{MPA} & \textbf{MAE} & \textbf{TAC} & \textbf{ACC} & \textbf{MPA} & \textbf{MAE} & \textbf{TAC} & \textbf{ACC} \\ 
	\midrule
	\multirow{ 2}{*}{\textbf{LSTM}} & High			& 	0.9922& 	1.2179& 	0.5640& \multirow{ 2}{*}{0.4183}
					  								& 	0.9856& 	4.2027& 	0.5872& \multirow{ 2}{*}{0.4163}	\\ 
									& Low			& 	0.9921& 	1.1985& 	0.5702& 
					  								& 	0.9851& 	4.1493& 	0.5711& 							\\ 
	\multirow{ 2}{*}{\textbf{GRU}} 	& High			& 	0.9929& 	1.1118& 	0.5640& \multirow{ 2}{*}{0.4340}	
													& 	0.9864& 	3.9909& 	0.5930& \multirow{ 2}{*}{0.4199} 	\\ 
									& Low			& 	0.9926& 	1.1193& 	0.5659& 	
													& 	0.9862& 	3.8426& 	0.5747& 							\\ 
	\multirow{ 2}{*}{\textbf{Transformer}} 	& High	& 	0.9906& 	1.4721& 	0.4743& \multirow{ 2}{*}{0.4295}	
													& 	0.9827& 	5.2046& 	0.4731& \multirow{ 2}{*}{0.3374} 	\\ 
											& Low	& 	0.9892& 	1.6502& 	0.5361& 
													& 	0.9830& 	4.8531& 	0.4616& 							\\ 
	\midrule
	\multicolumn{ 2}{c|}{\textbf{BASELINE}}			& 	0.9815& 		 & 		 & 	
					  								& 	0.9815& 		 & 		 & 									\\
	\bottomrule
	\bottomrule
	\end{tabular}
	% \caption{Result of our models}
\end{table}
As shown in Table \ref{tab:result}, the performance of our models on the test set is quantified. Incorporating VADER into the model has resulted in MPA surpassing that of the baseline model, which is DP-LSTM. 
Notably, the model's performance varies based on the stability of the stocks. For unstable stocks, the MAE significantly increases, and ACC decreases.
Both LSTM and GRU models exhibit similar performance metrics, whereas the Transformer model demonstrates slightly lower performance compared to the other two models. 
