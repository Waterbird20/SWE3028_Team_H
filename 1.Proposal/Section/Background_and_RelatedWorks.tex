\subsection{Background and Related Works}

\subsubsection{Stock Price Prediction}
Stock Price Prediction is the task of forecasting future stock prices based on historical data and various market indicators. 
It involves the use of statistical models and machine learning algorithms to analyze financial data and make predictions about a stock's future performance. 
Stock price trends are influenced by numerous factors, including interest rates, inflation rates, and financial news. 
To make accurate stock price predictions, one must leverage this diverse set of information. 
In the banking industry and financial services sector, teams of analysts are dedicated to scrutinizing, analyzing, and quantifying qualitative data from news sources. 
A substantial amount of information regarding stock trends is extracted from the extensive corpus of textual and quantitative data involved in such analysis.

\subsubsection{Research Trends}
Recent research trends in stock price prediction include advancements in deep learning-based regression models. 
These models often utilize Long Short-Term Memory (LSTM) networks and innovative validation techniques, such as walk-forward validation, to enhance their predictive capabilities.

In addition, some researchers have explored Particle Filter Recurrent Neural Networks (PF-RNNs), a new RNN family explicitly designed to model uncertainty within their internal structure. 
Unlike traditional RNNs that rely on a deterministic latent state vector, PF-RNNs maintain a latent state distribution approximated as a set of particles. 
To enable effective learning, researchers have introduced a fully differentiable particle filter algorithm that updates the PF-RNN latent state distribution based on Bayes' rule. 
Experimental results have shown that PF-RNNs can outperform conventional gated RNNs across various domains, 
including synthetic robot localization datasets and real-world sequence prediction tasks, which is stock price prediction.

Furthermore, recent studies have proposed novel approaches, such as the development of a sentiment-ARMA model, which combines the autoregressive moving average model (ARMA) with sentiment analysis of financial news articles. 
This model is integrated into an LSTM-based deep neural network consisting of three components: LSTM, VADER model, and a differential privacy (DP) mechanism. 
The proposed DP-LSTM scheme has demonstrated the potential to reduce prediction errors and enhance model robustness. Extensive experiments conducted on S\&P 500 stocks have indicated promising results, 
including a 0.32\% improvement in mean Mean Percentage Absolute Error (MPA) 
and a significant up to 65.79\% reduction in Mean Squared Error (MSE) for the prediction of the market index S\&P 500.

% \subsection{Background and Related Works}

% \subsubsection{Stock price prediction}
% Stock Price Prediction is the task of forecasting future stock prices based on historical data and various market indicators. 
% It involves using statistical models and machine learning algorithms to analyze financial data and make predictions about the future performance of a stock. 
% Stocks' trends are affected by a lot of factors such as interest rates, inflation rates and financial news.
% To predict stock prices accurately, one must use these variable information. In particular, in the
% banking industry and financial services, analysts' armies are dedicated to pouring over, analyzing,
% and attempting to quantify qualitative data from news. A large amount of stock trend information is
% extracted from the large amount of text and quantitative information that is involved in the analysis.

% \subsubsection{Research trend}
% We, then, augment the
% predictive power of our forecasting framework by building four deep learning-
% based regression models using long-and short-term memory (LSTM) networks
% with a novel approach of walk-forward validation.

% Particle Filter Recurrent Neural Networks (PF-RNNs),
% a new RNN family that explicitly models uncertainty in its in-
% ternal structure: while an RNN relies on a long, deterministic
% latent state vector, a PF-RNN maintains a latent state distribu-
% tion, approximated as a set of particles. For effective learning,
% we provide a fully differentiable particle filter algorithm that
% updates the PF-RNN latent state distribution according to the
% Bayes rule. Experiments demonstrate that the proposed PF-
% RNNs outperform the corresponding standard gated RNNs
% on a synthetic robot localization dataset and 10 real-world se-
% quence prediction datasets for text classification, stock price
% prediction, etc.

% First, based on the autoregressive moving average model (ARMA), a sentiment-
% ARMA is formulated by taking into consideration the information of financial
% news articles in the model. Then, an LSTM-based deep neural network is designed,
% which consists of three components: LSTM, VADER model and differential privacy
% (DP) mechanism. The proposed DP-LSTM scheme can reduce prediction errors
% and increase the robustness. Extensive experiments on S&P 500 stocks show that (i)
% the proposed DP-LSTM achieves 0.32\% improvement in mean MPA of prediction
% result, and (ii) for the prediction of the market index S&P 500, we achieve up to
% 65.79\% improvement in MSE.	





% https://paperswithcode.com/task/stock-price-prediction 
% 사이트를 참고하였습니다

% https://arxiv.org/ftp/arxiv/papers/2009/2009.10819.pdf
% LSTM을 이용하여 일일 단위 고가/저가를 예측한 모델입니다. 데이터 구성에는 일일 단위로 시가, 고가, 저가 , 종가, 거래량이 있네요. LSTM 모델 구현할 때 참고할 만 합니다.

% https://arxiv.org/pdf/1905.12885v2.pdf
% Particle Filtering을 적용한 LSTM과 GRU에 대해 설명하는 논문입니다. Particle Filtering은 비가우시안 통계를 따르는 데이터를 가우시안 통계로 근사하는 알고리즘인 것으로 이해했는데, 이를 적용하면 sequence 예측 성능이 개선되는 것 같아 고려할만한 LSTM과 GRU모델이 될 것 같습니다.

% https://arxiv.org/pdf/1912.10806v1.pdf
% 예측 모델로는 ARMA모델을 사용했는데, Sentiment Analysis 부분에서 참고할만한 내용이 있어서 선정하게 되었습니다. Sentiment Analysis 부분에서는 VADER를 사용했는데, 기업 공시 혹은 뉴스 정보도 예측에 사용한다면 도움이 될 것 같습니다.
% VADER 모델은 Sentiment Analysis을 수행하는 NLP모델입니다. 사전 학습이 되어 있어 따로 학습시킬 필요 없이 바로 사용할 수 있고, VADER 출력에 대해 적절히 임베딩만 해준다면 사용하기 간편할 것 같습니다.

% (논문이 거의 다 LSTM 뿐이네요... GRU도 조금만 보이고, CNN과 Transformer는 찾아볼 수 없었습니다ㅠㅠ 근데 우선 저정도로 하면 학습 방식 혹은 감성 분석 시 적용할 기술에 대해서는 참고할 양이 충분한 것 같습니다)