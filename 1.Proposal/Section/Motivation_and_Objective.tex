In today's world of capitalism, stocks offer an attractive way to make money, potentially leading to significant profits. 
Investors are always eager to predict what will happen to stock prices because their financial well-being depends on it. 
Investors must make prediction about future stock prices, even if they don't have a deep understanding of the stock market.

Predicting stock prices essentially involves trying to figure out the real value of a company. 
If you know a company's true value, you can decide whether its stock price will go up or down from its current level. 
However, determining a company's true value is a challenging task. 
Many factors, both within and beyond a company's control, influence stock prices. 
Some of these factors include global events and economic conditions. 

Historically, professional fund managers have played a crucial role in helping people invest wisely and profitably in the stock market. 
However, because fund managers are human, their predictions and strategies can be influenced by their personal biases. 
This has led to a growing interest in using machine learning, to predict stock prices. 
Machine learning is believed to offer a more objective and unbiased approach to forecasting.

These days, numerous services utilize machine learning to predict stock prices. 
For instance, in Korea, a service known as 'IAM CHART' employs machine learning for this purpose. 
However, this service comes at a cost and is not open source. 
With a monthly fee of \$150, it is beyond the means of many potential users.

In light of this, our intention is to harness machine learning techniques to offer a more accessible solution, 
thereby contributing to the realms of financial analysis and investment decision-making at a lower price point. 
Our project's primary objective is to construct a predictive model for the stock prices of SAMSUNG Electronics.
